%%%%%%%%%%%%%%%%%%%%%%%%%%%%%%%%%%%%%%%%%
% Simple Sectioned Essay Template
% LaTeX Template
%
% This template has been downloaded from:
% http://www.latextemplates.com
%
% Note:
% The \lipsum[#] commands throughout this template generate dummy text
% to fill the template out. These commands should all be removed when 
% writing essay content.
%
%%%%%%%%%%%%%%%%%%%%%%%%%%%%%%%%%%%%%%%%%

%----------------------------------------------------------------------------------------
%	PACKAGES AND OTHER DOCUMENT CONFIGURATIONS
%----------------------------------------------------------------------------------------

\documentclass[12pt]{article} % Default font size is 12pt, it can be changed here

\usepackage{geometry} % Required to change the page size to A4
\geometry{a4paper} % Set the page size to be A4 as opposed to the default US Letter

\usepackage{graphicx} % Required for including pictures

\usepackage{float} % Allows putting an [H] in \begin{figure} to specify the exact location of the figure
\usepackage{wrapfig} % Allows in-line images such as the example fish picture

\usepackage{lipsum} % Used for inserting dummy 'Lorem ipsum' text into the template

\usepackage{hyperref}


\linespread{1.2} % Line spacing

%\setlength\parindent{0pt} % Uncomment to remove all indentation from paragraphs

\graphicspath{{Pictures/}} % Specifies the directory where pictures are stored

\usepackage{listings}
\usepackage{color}

\definecolor{dkgreen}{rgb}{0,0.6,0}
\definecolor{gray}{rgb}{0.5,0.5,0.5}
\definecolor{mauve}{rgb}{0.58,0,0.82}


\definecolor{Maroon}{rgb}{0.5,0,0}
\definecolor{darkgreen}{rgb}{0,0.5,0}

\lstset{frame=tb,
	backgroundcolor=\color[rgb]{0.9,0.9,0.9},
	language=Java,
	aboveskip=3mm,
	belowskip=3mm,
	showstringspaces=false,
	columns=flexible,
	basicstyle={\small\ttfamily},
	% Line numbers
	xleftmargin={0.75cm},
	numbers=left,
	stepnumber=1,
	firstnumber=1,
	numberfirstline=true,
	numberstyle=\tiny\color{gray},
	keywordstyle=\color{blue},
	commentstyle=\color{dkgreen},
	stringstyle=\color{mauve},
	breaklines=true,
	breakatwhitespace=true,
	tabsize=3
}


\lstdefinelanguage{XML_android}
{
	alsoletter=-,
	basicstyle=\ttfamily\footnotesize,
	morestring=[b]",
	moredelim=*[s][\color{Maroon}]{<}{\ },
	moredelim=[s][\color{Maroon}]{</}{>},
	moredelim=[l][\color{Maroon}]{/>},
	moredelim=[l][\color{Maroon}]{>},
	morecomment=[s]{<?}{?>},
	morecomment=[s]{<!--}{-->},
	morecomment=[s]{<!}{>},
	commentstyle=\color{darkgreen},
	stringstyle=\color{blue},
	identifierstyle=\color{red}
}




\begin{document}

%----------------------------------------------------------------------------------------
%	TITLE PAGE
%----------------------------------------------------------------------------------------

\begin{titlepage}

\newcommand{\HRule}{\rule{\linewidth}{0.5mm}} % Defines a new command for the horizontal lines, change thickness here

\center % Center everything on the page

\HRule \\[0.4cm]
{ \huge \bfseries Create a RecyclerView}\\[0.4cm] % Title of your document
\HRule \\[1.5cm]

\begin{minipage}{0.4\textwidth}
\end{minipage}\\[4cm]

{\large \today}\\[3cm] % Date, change the \today to a set date if you want to be precise

%\includegraphics{Logo}\\[1cm] % Include a department/university logo - this will require the graphicx package

\vfill % Fill the rest of the page with whitespace

\end{titlepage}

%----------------------------------------------------------------------------------------
%	TABLE OF CONTENTS
%----------------------------------------------------------------------------------------

\tableofcontents % Include a table of contents

\newpage % Begins the essay on a new page instead of on the same page as the table of contents 

%----------------------------------------------------------------------------------------
%	INTRODUCTION
%----------------------------------------------------------------------------------------

\section{Introduction} % Major section

How to create a RecyclerView.

%------------------------------------------------

\section{Add RecyclerView dependency}
Add RecyclerView dependency to the app grandle:

\begin{lstlisting}[%
language=XML_android]
compile 'com.android.support:recyclerview-v7:25.1.0'
\end{lstlisting}

\section{Create a layout with RecyclerView}

\begin{lstlisting}[%
language=XML_android]
<FrameLayout xmlns:android="http://schemas.android.com/apk/res/android"
	android:layout_width="match_parent"
	android:layout_height="match_parent">

	<android.support.v7.widget.RecyclerView
		android:layout_width="match_parent"
		android:layout_height="match_parent"
		android:id="@+id/recyclerview_forecast"/>

	<TextView
		android:id="@+id/tv_weather_data"
		android:layout_width="wrap_content"
		android:layout_height="wrap_content"
		android:padding="16dp"
		android:textSize="20sp" />

	<TextView
		android:id="@+id/tv_error_message_display"
		android:layout_width="wrap_content"
		android:layout_height="wrap_content"
		android:padding="16dp"
		android:text="@string/error_message"
		android:textSize="20sp"
		android:visibility="invisible" />

	<ProgressBar
		android:id="@+id/pb_loading_indicator"
		android:layout_height="42dp"
		android:layout_width="42dp"
		android:layout_gravity="center"
		android:visibility="invisible" />

</FrameLayout>
\end{lstlisting}

\section{Create item Layout}
Create a generic Layout for items that will be storaged in the RecyclerView

Example:

\begin{lstlisting}[%
language=XML_android]
<?xml version="1.0" encoding="utf-8"?>
<LinearLayout xmlns:android="http://schemas.android.com/apk/res/android"
android:layout_width="match_parent"
android:layout_height="wrap_content"
android:orientation="vertical">

<TextView
android:layout_width="wrap_content"
android:layout_height="wrap_content"
android:id="@+id/tv_weather_data"
android:textSize="22sp"
android:padding="16dp"/>

<View
android:layout_width="match_parent"
android:layout_height="1dp"
android:background="#dadada"
android:layout_marginLeft="8dp"
android:layout_marginRight="8dp">

</View>

</LinearLayout>
\end{lstlisting}

\section{Create the adapter}

\begin{enumerate}
	\item Create a class for the adapter.
	\item Extend it to extends RecyclerView.Adapter<ForecastAdapter.ForecastAdapterViewHolder>.
	\item Create an array where you will storage the data for each item.
	\item Create an empty constructor.
	\item Create a class inside your class which extends RecyclerView.ViewHolder.
	\item Inside your RecyclerView.ViewHolder class create the variables for your TextView or whatever you have in your RecyclerView Layout.
	\item Inside your RecyclerView.ViewHolder class create a constructor with View as a parameter.
	\item Inside your RecyclerView.ViewHolder class constructor get the references of your RecyclerView Layout TextViews.
	\item Inside your adapter class override onCreateViewHolder and inflate your RecyclerView Layout into a view.
	\item Return ForecastAdapterViewHolder with the inflated view.
	\item Override onBindViewHolder and set the text of the TextView.
	\item Override getItemCount and return 0 if the array which we created in the beginning is null or the length if not.
	\item create a method to set data in your array.
\end{enumerate}

\begin{lstlisting}[language=Java]
/*
* Copyright (C) 2016 The Android Open Source Project
*
* Licensed under the Apache License, Version 2.0 (the "License");
* you may not use this file except in compliance with the License.
* You may obtain a copy of the License at
*
*      http://www.apache.org/licenses/LICENSE-2.0
*
* Unless required by applicable law or agreed to in writing, software
* distributed under the License is distributed on an "AS IS" BASIS,
* WITHOUT WARRANTIES OR CONDITIONS OF ANY KIND, either express or implied.
* See the License for the specific language governing permissions and
* limitations under the License.
*/
package com.example.android.sunshine;

import android.content.Context;
import android.support.v7.widget.RecyclerView;
import android.view.LayoutInflater;
import android.view.View;
import android.view.ViewGroup;
import android.widget.TextView;

/**
* {@link ForecastAdapter} exposes a list of weather forecasts to a
* {@link android.support.v7.widget.RecyclerView}
*/
public class ForecastAdapter extends RecyclerView.Adapter<ForecastAdapter.ForecastAdapterViewHolder> {

	private String[] mWeatherData;
	
	public ForecastAdapter() {
		
	}
	
	/**
	* Cache of the children views for a forecast list item.
	*/
	public class ForecastAdapterViewHolder extends RecyclerView.ViewHolder {
		
		// Within ForecastAdapterViewHolder ///////////////////////////////////////////////////////
		public final TextView mWeatherTextView;
		
		public ForecastAdapterViewHolder(View view) {
			super(view);
			mWeatherTextView = (TextView) view.findViewById(R.id.tv_weather_data);
		}
		// Within ForecastAdapterViewHolder ///////////////////////////////////////////////////////
	}
	
	/**
	* This gets called when each new ViewHolder is created. This happens when the RecyclerView
	* is laid out. Enough ViewHolders will be created to fill the screen and allow for scrolling.
	*
	* @param viewGroup The ViewGroup that these ViewHolders are contained within.
	* @param viewType  If your RecyclerView has more than one type of item (which ours doesn't) you
	*                  can use this viewType integer to provide a different layout. See
	*                  {@link android.support.v7.widget.RecyclerView.Adapter#getItemViewType(int)}
	*                  for more details.
	* @return A new ForecastAdapterViewHolder that holds the View for each list item
	*/
	@Override
	public ForecastAdapterViewHolder onCreateViewHolder(ViewGroup viewGroup, int viewType) {
		Context context = viewGroup.getContext();
		int layoutIdForListItem = R.layout.forecast_list_item;
		LayoutInflater inflater = LayoutInflater.from(context);
		boolean shouldAttachToParentImmediately = false;
		
		View view = inflater.inflate(layoutIdForListItem, viewGroup, shouldAttachToParentImmediately);
		return new ForecastAdapterViewHolder(view);
	}
	
	/**
	* OnBindViewHolder is called by the RecyclerView to display the data at the specified
	* position. In this method, we update the contents of the ViewHolder to display the weather
	* details for this particular position, using the "position" argument that is conveniently
	* passed into us.
	*
	* @param forecastAdapterViewHolder The ViewHolder which should be updated to represent the
	*                                  contents of the item at the given position in the data set.
	* @param position                  The position of the item within the adapter's data set.
	*/
	@Override
	public void onBindViewHolder(ForecastAdapterViewHolder forecastAdapterViewHolder, int position) {
		String weatherForThisDay = mWeatherData[position];
		forecastAdapterViewHolder.mWeatherTextView.setText(weatherForThisDay);
	}
	
	/**
	* This method simply returns the number of items to display. It is used behind the scenes
	* to help layout our Views and for animations.
	*
	* @return The number of items available in our forecast
	*/
	@Override
	public int getItemCount() {
		if (null == mWeatherData) return 0;
		return mWeatherData.length;
	}
	
	/**
	* This method is used to set the weather forecast on a ForecastAdapter if we've already
	* created one. This is handy when we get new data from the web but don't want to create a
	* new ForecastAdapter to display it.
	*
	* @param weatherData The new weather data to be displayed.
	*/
	public void setWeatherData(String[] weatherData) {
		mWeatherData = weatherData;
		notifyDataSetChanged();
	}
}

\end{lstlisting}



\section{Edit your MainActivity}

\begin{enumerate}
	\item Add a RecyclerView variable.
	\item Add your new class adapter variable.
	\item Get a reference from your RecyclerView in the variable.
	\item Create a LinearLayoutManager and set context, oritentation and shouldReverseLayout.
	\item Set your LinearLayoutManager in your RecyclerView
	\item set setHasFixedSize in your RecyclerView if you want to keep all the components with the same size.
	\item Initialize your new class adapter variable.
	\item Use setAdapter to set your adapter to you RecyclerView.
	\item Set the data to your adapter.
	
\end{enumerate}

Example:

\begin{lstlisting}[language=Java]

/*
* Copyright (C) 2016 The Android Open Source Project
*
* Licensed under the Apache License, Version 2.0 (the "License");
* you may not use this file except in compliance with the License.
* You may obtain a copy of the License at
*
*      http://www.apache.org/licenses/LICENSE-2.0
*
* Unless required by applicable law or agreed to in writing, software
* distributed under the License is distributed on an "AS IS" BASIS,
* WITHOUT WARRANTIES OR CONDITIONS OF ANY KIND, either express or implied.
* See the License for the specific language governing permissions and
* limitations under the License.
*/
package com.example.android.sunshine;

import android.os.AsyncTask;
import android.os.Bundle;
import android.support.v7.app.AppCompatActivity;
import android.support.v7.widget.LinearLayoutManager;
import android.support.v7.widget.RecyclerView;
import android.view.Menu;
import android.view.MenuInflater;
import android.view.MenuItem;
import android.view.View;
import android.widget.ProgressBar;
import android.widget.TextView;

import com.example.android.sunshine.data.SunshinePreferences;
import com.example.android.sunshine.utilities.NetworkUtils;
import com.example.android.sunshine.utilities.OpenWeatherJsonUtils;

import java.net.URL;

public class MainActivity extends AppCompatActivity {
	
	private RecyclerView mRecyclerView;
	
	private ForecastAdapter mForecastAdapter;
	
	private TextView mErrorMessageDisplay;
	
	private ProgressBar mLoadingIndicator;
	
	@Override
	protected void onCreate(Bundle savedInstanceState) {
		super.onCreate(savedInstanceState);
		setContentView(R.layout.activity_forecast);
		

		/*
		* Using findViewById, we get a reference to our RecyclerView from xml. This allows us to
		* do things like set the adapter of the RecyclerView and toggle the visibility.
		*/
		mRecyclerView = (RecyclerView) findViewById(R.id.recyclerview_forecast);
		
		/* This TextView is used to display errors and will be hidden if there are no errors */
		mErrorMessageDisplay = (TextView) findViewById(R.id.tv_error_message_display);
		
		/*
		* LinearLayoutManager can support HORIZONTAL or VERTICAL orientations. The reverse layout
		* parameter is useful mostly for HORIZONTAL layouts that should reverse for right to left
		* languages.
		*/
		LinearLayoutManager layoutManager
		= new LinearLayoutManager(this, LinearLayoutManager.VERTICAL, false);
		
		mRecyclerView.setLayoutManager(layoutManager);
		
		/*
		* Use this setting to improve performance if you know that changes in content do not
		* change the child layout size in the RecyclerView
		*/
		mRecyclerView.setHasFixedSize(true);
		
		/*
		* The ForecastAdapter is responsible for linking our weather data with the Views that
		* will end up displaying our weather data.
		*/
		mForecastAdapter = new ForecastAdapter();

		/* Setting the adapter attaches it to the RecyclerView in our layout. */
		mRecyclerView.setAdapter(mForecastAdapter);
		
		/*
		* The ProgressBar that will indicate to the user that we are loading data. It will be
		* hidden when no data is loading.
		*
		* Please note: This so called "ProgressBar" isn't a bar by default. It is more of a
		* circle. We didn't make the rules (or the names of Views), we just follow them.
		*/
		mLoadingIndicator = (ProgressBar) findViewById(R.id.pb_loading_indicator);
		
		/* Once all of our views are setup, we can load the weather data. */
		loadWeatherData();
	}
	
	/**
	* This method will get the user's preferred location for weather, and then tell some
	* background method to get the weather data in the background.
	*/
	private void loadWeatherData() {
		showWeatherDataView();
		
		String location = SunshinePreferences.getPreferredWeatherLocation(this);
		new FetchWeatherTask().execute(location);
	}
	
	/**
	* This method will make the View for the weather data visible and
	* hide the error message.
	* <p>
	* Since it is okay to redundantly set the visibility of a View, we don't
	* need to check whether each view is currently visible or invisible.
	*/
	private void showWeatherDataView() {
		/* First, make sure the error is invisible */
		mErrorMessageDisplay.setVisibility(View.INVISIBLE);

		/* Then, make sure the weather data is visible */
		mRecyclerView.setVisibility(View.VISIBLE);
	}
	
	/**
	* This method will make the error message visible and hide the weather
	* View.
	* <p>
	* Since it is okay to redundantly set the visibility of a View, we don't
	* need to check whether each view is currently visible or invisible.
	*/
	private void showErrorMessage() {
		/* First, hide the currently visible data */
		mRecyclerView.setVisibility(View.INVISIBLE);
		/* Then, show the error */
		mErrorMessageDisplay.setVisibility(View.VISIBLE);
	}
	
	public class FetchWeatherTask extends AsyncTask<String, Void, String[]> {
		
		@Override
		protected void onPreExecute() {
			super.onPreExecute();
			mLoadingIndicator.setVisibility(View.VISIBLE);
		}
		
		@Override
		protected String[] doInBackground(String... params) {
			
			/* If there's no zip code, there's nothing to look up. */
			if (params.length == 0) {
				return null;
			}
			
			String location = params[0];
			URL weatherRequestUrl = NetworkUtils.buildUrl(location);
			
			try {
				String jsonWeatherResponse = NetworkUtils
				.getResponseFromHttpUrl(weatherRequestUrl);
				
				String[] simpleJsonWeatherData = OpenWeatherJsonUtils
				.getSimpleWeatherStringsFromJson(MainActivity.this, jsonWeatherResponse);
				
				return simpleJsonWeatherData;
				
			} catch (Exception e) {
				e.printStackTrace();
				return null;
			}
		}
		
		@Override
		protected void onPostExecute(String[] weatherData) {
			mLoadingIndicator.setVisibility(View.INVISIBLE);
			if (weatherData != null) {
				showWeatherDataView();
				// COMPLETED (45) Instead of iterating through every string, use mForecastAdapter.setWeatherData and pass in the weather data
				mForecastAdapter.setWeatherData(weatherData);
			} else {
				showErrorMessage();
			}
		}
	}
	
	@Override
	public boolean onCreateOptionsMenu(Menu menu) {
		/* Use AppCompatActivity's method getMenuInflater to get a handle on the menu inflater */
		MenuInflater inflater = getMenuInflater();
		/* Use the inflater's inflate method to inflate our menu layout to this menu */
		inflater.inflate(R.menu.forecast, menu);
		/* Return true so that the menu is displayed in the Toolbar */
		return true;
	}
	
	@Override
	public boolean onOptionsItemSelected(MenuItem item) {
		int id = item.getItemId();
		
		if (id == R.id.action_refresh) {
			mForecastAdapter.setWeatherData(null);
			loadWeatherData();
			return true;
		}
		
		return super.onOptionsItemSelected(item);
	}
}
\end{lstlisting}
\href{https://developer.android.com/reference/android/support/v7/widget/RecyclerView.html}{RecyclerView Resource}

\end{document}
