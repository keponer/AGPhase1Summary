%%%%%%%%%%%%%%%%%%%%%%%%%%%%%%%%%%%%%%%%%
% Simple Sectioned Essay Template
% LaTeX Template
%
% This template has been downloaded from:
% http://www.latextemplates.com
%
% Note:
% The \lipsum[#] commands throughout this template generate dummy text
% to fill the template out. These commands should all be removed when 
% writing essay content.
%
%%%%%%%%%%%%%%%%%%%%%%%%%%%%%%%%%%%%%%%%%

%----------------------------------------------------------------------------------------
%	PACKAGES AND OTHER DOCUMENT CONFIGURATIONS
%----------------------------------------------------------------------------------------

\documentclass[12pt]{article} % Default font size is 12pt, it can be changed here

\usepackage{geometry} % Required to change the page size to A4
\geometry{a4paper} % Set the page size to be A4 as opposed to the default US Letter

\usepackage{graphicx} % Required for including pictures

\usepackage{float} % Allows putting an [H] in \begin{figure} to specify the exact location of the figure
\usepackage{wrapfig} % Allows in-line images such as the example fish picture

\usepackage{lipsum} % Used for inserting dummy 'Lorem ipsum' text into the template

\usepackage{hyperref}


\linespread{1.2} % Line spacing

%\setlength\parindent{0pt} % Uncomment to remove all indentation from paragraphs

\graphicspath{{Pictures/}} % Specifies the directory where pictures are stored

\usepackage{listings}
\usepackage{color}

\definecolor{dkgreen}{rgb}{0,0.6,0}
\definecolor{gray}{rgb}{0.5,0.5,0.5}
\definecolor{mauve}{rgb}{0.58,0,0.82}


\definecolor{Maroon}{rgb}{0.5,0,0}
\definecolor{darkgreen}{rgb}{0,0.5,0}

\lstset{frame=tb,
	backgroundcolor=\color[rgb]{0.9,0.9,0.9},
	language=Java,
	aboveskip=3mm,
	belowskip=3mm,
	showstringspaces=false,
	columns=flexible,
	basicstyle={\small\ttfamily},
	% Line numbers
	xleftmargin={0.75cm},
	numbers=left,
	stepnumber=1,
	firstnumber=1,
	numberfirstline=true,
	numberstyle=\tiny\color{gray},
	keywordstyle=\color{blue},
	commentstyle=\color{dkgreen},
	stringstyle=\color{mauve},
	breaklines=true,
	breakatwhitespace=true,
	tabsize=3
}


\lstdefinelanguage{XML_android}
{
	alsoletter=-,
	basicstyle=\ttfamily\footnotesize,
	morestring=[b]",
	moredelim=*[s][\color{Maroon}]{<}{\ },
	moredelim=[s][\color{Maroon}]{</}{>},
	moredelim=[l][\color{Maroon}]{/>},
	moredelim=[l][\color{Maroon}]{>},
	morecomment=[s]{<?}{?>},
	morecomment=[s]{<!--}{-->},
	morecomment=[s]{<!}{>},
	commentstyle=\color{darkgreen},
	stringstyle=\color{blue},
	identifierstyle=\color{red}
}




\begin{document}

%----------------------------------------------------------------------------------------
%	TITLE PAGE
%----------------------------------------------------------------------------------------

\begin{titlepage}

\newcommand{\HRule}{\rule{\linewidth}{0.5mm}} % Defines a new command for the horizontal lines, change thickness here

\center % Center everything on the page

\HRule \\[0.4cm]
{ \huge \bfseries Create a ClickHandler for RecyclerView items }\\[0.4cm] % Title of your document
\HRule \\[1.5cm]

\begin{minipage}{0.4\textwidth}
\end{minipage}\\[4cm]

{\large \today}\\[3cm] % Date, change the \today to a set date if you want to be precise

%\includegraphics{Logo}\\[1cm] % Include a department/university logo - this will require the graphicx package

\vfill % Fill the rest of the page with whitespace

\end{titlepage}

%----------------------------------------------------------------------------------------
%	TABLE OF CONTENTS
%----------------------------------------------------------------------------------------

\tableofcontents % Include a table of contents

\newpage % Begins the essay on a new page instead of on the same page as the table of contents 

%----------------------------------------------------------------------------------------
%	INTRODUCTION
%----------------------------------------------------------------------------------------

\section{Introduction} % Major section

How to add a ClickHandler for your items in a RecyclerView Layout.

%------------------------------------------------

\section{Edit your Adapter}

\begin{enumerate}
	\item Add an Interface.
	\item Within that interface, define a void method that access a variable as a parameter.
	\item Create a final private yourInterface attribute.
	\item  Add yourInterface as a parameter to the constructor and store it in the attribute created before.
	\item Implement OnClickListener in your class AdapterViewHolder class.
	\item  Override onClick, passing what you want to yourInterface via its method.
	\item Call setOnClickListener on the view passed into the constructor
\end{enumerate}

\begin{lstlisting}[language=Java]
private final ForecastAdapterOnClickHandler mClickHandler;

public interface ForecastAdapterOnClickHandler{
	public void onClick(String s);
}

public ForecastAdapter(ForecastAdapterOnClickHandler mClickHandler) {
	this.mClickHandler = mClickHandler;
	
}

/**
* Cache of the children views for a forecast list item.
*/
public class ForecastAdapterViewHolder extends RecyclerView.ViewHolder implements View.OnClickListener{
	public final TextView mWeatherTextView;
	
	public ForecastAdapterViewHolder(View view) {
		super(view);
		mWeatherTextView = (TextView) view.findViewById(R.id.tv_weather_data);
		view.setOnClickListener(this);
		
	}
	
	@Override
	public void onClick(View v) {
		int adapterPosition = getAdapterPosition();
		String weatherForTodaz = mWeatherData[adapterPosition];
		mClickHandler.onClick(weatherForTodaz);
	}
	
	
}

\end{lstlisting}

\section{Edit your Activity}

\begin{enumerate}
	\item Implement your interface from the MainActivity.
	\item Override interface method.
	\item Pass in 'this' as your Adapter constructor.
\end{enumerate}

\begin{lstlisting}[language=Java]
public class MainActivity extends AppCompatActivity implements ForecastAdapter.ForecastAdapterOnClickHandler{
	
	private RecyclerView mRecyclerView;
	private ForecastAdapter mForecastAdapter;
	
	private TextView mErrorMessageDisplay;
	
	private ProgressBar mLoadingIndicator;
	
	@Override
	protected void onCreate(Bundle savedInstanceState) {
		super.onCreate(savedInstanceState);
		setContentView(R.layout.activity_forecast);
		
		/*
		* Using findViewById, we get a reference to our RecyclerView from xml. This allows us to
		* do things like set the adapter of the RecyclerView and toggle the visibility.
		*/
		mRecyclerView = (RecyclerView) findViewById(R.id.recyclerview_forecast);
		
		/* This TextView is used to display errors and will be hidden if there are no errors */
		mErrorMessageDisplay = (TextView) findViewById(R.id.tv_error_message_display);
		
		/*
		* LinearLayoutManager can support HORIZONTAL or VERTICAL orientations. The reverse layout
		* parameter is useful mostly for HORIZONTAL layouts that should reverse for right to left
		* languages.
		*/
		LinearLayoutManager layoutManager
		= new LinearLayoutManager(this, LinearLayoutManager.VERTICAL, false);
		
		mRecyclerView.setLayoutManager(layoutManager);
		
		/*
		* Use this setting to improve performance if you know that changes in content do not
		* change the child layout size in the RecyclerView
		*/
		mRecyclerView.setHasFixedSize(true);

		/*
		* The ForecastAdapter is responsible for linking our weather data with the Views that
		* will end up displaying our weather data.
		*/
		mForecastAdapter = new ForecastAdapter(this);
		
		/* Setting the adapter attaches it to the RecyclerView in our layout. */
		mRecyclerView.setAdapter(mForecastAdapter);
		
		/*
		* The ProgressBar that will indicate to the user that we are loading data. It will be
		* hidden when no data is loading.
		*
		* Please note: This so called "ProgressBar" isn't a bar by default. It is more of a
		* circle. We didn't make the rules (or the names of Views), we just follow them.
		*/
		mLoadingIndicator = (ProgressBar) findViewById(R.id.pb_loading_indicator);
		
		/* Once all of our views are setup, we can load the weather data. */
		loadWeatherData();
	}
	
	/**
	* This method will get the user's preferred location for weather, and then tell some
	* background method to get the weather data in the background.
	*/
	private void loadWeatherData() {
		showWeatherDataView();
		
		String location = SunshinePreferences.getPreferredWeatherLocation(this);
		new FetchWeatherTask().execute(location);
	}
	
	@Override
	public void onClick(String s) {
		Toast.makeText(this,s,Toast.LENGTH_LONG).show();
	}
	
\end{lstlisting}

\end{document}
