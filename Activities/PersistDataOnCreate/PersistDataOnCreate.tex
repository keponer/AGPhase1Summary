%%%%%%%%%%%%%%%%%%%%%%%%%%%%%%%%%%%%%%%%%
% Simple Sectioned Essay Template
% LaTeX Template
%
% This template has been downloaded from:
% http://www.latextemplates.com
%
% Note:
% The \lipsum[#] commands throughout this template generate dummy text
% to fill the template out. These commands should all be removed when 
% writing essay content.
%
%%%%%%%%%%%%%%%%%%%%%%%%%%%%%%%%%%%%%%%%%

%----------------------------------------------------------------------------------------
%	PACKAGES AND OTHER DOCUMENT CONFIGURATIONS
%----------------------------------------------------------------------------------------

\documentclass[12pt]{article} % Default font size is 12pt, it can be changed here

\usepackage{geometry} % Required to change the page size to A4
\geometry{a4paper} % Set the page size to be A4 as opposed to the default US Letter

\usepackage{graphicx} % Required for including pictures

\usepackage{float} % Allows putting an [H] in \begin{figure} to specify the exact location of the figure
\usepackage{wrapfig} % Allows in-line images such as the example fish picture

\usepackage{lipsum} % Used for inserting dummy 'Lorem ipsum' text into the template

\usepackage{hyperref}


\linespread{1.2} % Line spacing

%\setlength\parindent{0pt} % Uncomment to remove all indentation from paragraphs

\graphicspath{{Pictures/}} % Specifies the directory where pictures are stored

\usepackage{listings}
\usepackage{color}

\definecolor{dkgreen}{rgb}{0,0.6,0}
\definecolor{gray}{rgb}{0.5,0.5,0.5}
\definecolor{mauve}{rgb}{0.58,0,0.82}


\definecolor{Maroon}{rgb}{0.5,0,0}
\definecolor{darkgreen}{rgb}{0,0.5,0}

\lstset{frame=tb,
	backgroundcolor=\color[rgb]{0.9,0.9,0.9},
	language=Java,
	aboveskip=3mm,
	belowskip=3mm,
	showstringspaces=false,
	columns=flexible,
	basicstyle={\small\ttfamily},
	% Line numbers
	xleftmargin={0.75cm},
	numbers=left,
	stepnumber=1,
	firstnumber=1,
	numberfirstline=true,
	numberstyle=\tiny\color{gray},
	keywordstyle=\color{blue},
	commentstyle=\color{dkgreen},
	stringstyle=\color{mauve},
	breaklines=true,
	breakatwhitespace=true,
	tabsize=3
}


\lstdefinelanguage{XML_android}
{
	alsoletter=-,
	basicstyle=\ttfamily\footnotesize,
	morestring=[b]",
	moredelim=*[s][\color{Maroon}]{<}{\ },
	moredelim=[s][\color{Maroon}]{</}{>},
	moredelim=[l][\color{Maroon}]{/>},
	moredelim=[l][\color{Maroon}]{>},
	morecomment=[s]{<?}{?>},
	morecomment=[s]{<!--}{-->},
	morecomment=[s]{<!}{>},
	commentstyle=\color{darkgreen},
	stringstyle=\color{blue},
	identifierstyle=\color{red}
}




\begin{document}

%----------------------------------------------------------------------------------------
%	TITLE PAGE
%----------------------------------------------------------------------------------------

\begin{titlepage}

\newcommand{\HRule}{\rule{\linewidth}{0.5mm}} % Defines a new command for the horizontal lines, change thickness here

\center % Center everything on the page

\HRule \\[0.4cm]
{ \huge \bfseries Persist data when the screen is flipped}\\[0.4cm] % Title of your document
\HRule \\[1.5cm]

\begin{minipage}{0.4\textwidth}
\end{minipage}\\[4cm]

{\large \today}\\[3cm] % Date, change the \today to a set date if you want to be precise

%\includegraphics{Logo}\\[1cm] % Include a department/university logo - this will require the graphicx package

\vfill % Fill the rest of the page with whitespace

\end{titlepage}

%----------------------------------------------------------------------------------------
%	TABLE OF CONTENTS
%----------------------------------------------------------------------------------------

\tableofcontents % Include a table of contents

\newpage % Begins the essay on a new page instead of on the same page as the table of contents 

%----------------------------------------------------------------------------------------
%	INTRODUCTION
%----------------------------------------------------------------------------------------

\section{Introduction} % Major section

How to persista data when you flip the screen of your phone. In this example we will do it with a TextView.

%------------------------------------------------

\section{Create the methods}

\begin{enumerate}
	\item Create a key String it will be a constant to identify your bundle.
	\item Override onSaveInstanceState.
	\item Call super.onSaveInstanceState.
	\item Put the text from the TextView in the outState bundle with bundle.putString(key, value).
	\item Check in your onCreate() if  savedInstanceState is not null and contains your contant, set that text on our TextView.
\end{enumerate}

\begin{lstlisting}[language=Java]
public class MainActivity extends AppCompatActivity {
	
	private static final String TAG = MainActivity.class.getSimpleName();
	
	// COMPLETED (1) Create a key String called LIFECYCLE_CALLBACKS_TEXT_KEY
	/*
	* This constant String will be used to store the content of the TextView used to display the
	* list of callbacks. The reason we are storing the contents of the TextView is so that you can
	* see the entire set of callbacks as they are called.
	*/
	private static final String LIFECYCLE_CALLBACKS_TEXT_KEY = "callbacks";
	
	/* Constant values for the names of each respective lifecycle callback */
	private static final String ON_CREATE = "onCreate";
	private static final String ON_START = "onStart";
	private static final String ON_RESUME = "onResume";
	private static final String ON_PAUSE = "onPause";
	private static final String ON_STOP = "onStop";
	private static final String ON_RESTART = "onRestart";
	private static final String ON_DESTROY = "onDestroy";
	private static final String ON_SAVE_INSTANCE_STATE = "onSaveInstanceState";
	
	/*
	* This TextView will contain a running log of every lifecycle callback method called from this
	* Activity. This TextView can be reset to its default state by clicking the Button labeled
	* "Reset Log"
	*/
	private TextView mLifecycleDisplay;
	
	/**
	* Called when the activity is first created. This is where you should do all of your normal
	* static set up: create views, bind data to lists, etc.
	*
	* Always followed by onStart().
	*
	* @param savedInstanceState The Activity's previously frozen state, if there was one.
	*/
	@Override
	protected void onCreate(Bundle savedInstanceState) {
		super.onCreate(savedInstanceState);
		setContentView(R.layout.activity_main);
		
		mLifecycleDisplay = (TextView) findViewById(R.id.tv_lifecycle_events_display);
		
		/*
		* If savedInstanceState is not null, that means our Activity is not being started for the
		* first time. Even if the savedInstanceState is not null, it is smart to check if the
		* bundle contains the key we are looking for. In our case, the key we are looking for maps
		* to the contents of the TextView that displays our list of callbacks. If the bundle
		* contains that key, we set the contents of the TextView accordingly.
		*/
		if (savedInstanceState != null) {
			if (savedInstanceState.containsKey(LIFECYCLE_CALLBACKS_TEXT_KEY)) {
				String allPreviousLifecycleCallbacks = savedInstanceState
				.getString(LIFECYCLE_CALLBACKS_TEXT_KEY);
				mLifecycleDisplay.setText(allPreviousLifecycleCallbacks);
			}
		}
		
		logAndAppend(ON_CREATE);
	}
	
	/**
	* Called when the activity is becoming visible to the user.
	*
	* Followed by onResume() if the activity comes to the foreground, or onStop() if it becomes
	* hidden.
	*/
	@Override
	protected void onStart() {
		super.onStart();
		
		logAndAppend(ON_START);
	}
	
	/**
	* Called when the activity will start interacting with the user. At this point your activity
	* is at the top of the activity stack, with user input going to it.
	*
	* Always followed by onPause().
	*/
	@Override
	protected void onResume() {
		super.onResume();
		
		logAndAppend(ON_RESUME);
	}
	
	/**
	* Called when the system is about to start resuming a previous activity. This is typically
	* used to commit unsaved changes to persistent data, stop animations and other things that may
	* be consuming CPU, etc. Implementations of this method must be very quick because the next
	* activity will not be resumed until this method returns.
	*
	* Followed by either onResume() if the activity returns back to the front, or onStop() if it
	* becomes invisible to the user.
	*/
	@Override
	protected void onPause() {
		super.onPause();
		
		logAndAppend(ON_PAUSE);
	}
	
	/**
	* Called when the activity is no longer visible to the user, because another activity has been
	* resumed and is covering this one. This may happen either because a new activity is being
	* started, an existing one is being brought in front of this one, or this one is being
	* destroyed.
	*
	* Followed by either onRestart() if this activity is coming back to interact with the user, or
	* onDestroy() if this activity is going away.
	*/
	@Override
	protected void onStop() {
		super.onStop();
		
		logAndAppend(ON_STOP);
	}
	
	/**
	* Called after your activity has been stopped, prior to it being started again.
	*
	* Always followed by onStart()
	*/
	@Override
	protected void onRestart() {
		super.onRestart();
		
		logAndAppend(ON_RESTART);
	}
	
	/**
	* The final call you receive before your activity is destroyed. This can happen either because
	* the activity is finishing (someone called finish() on it, or because the system is
	* temporarily destroying this instance of the activity to save space. You can distinguish
	* between these two scenarios with the isFinishing() method.
	*/
	@Override
	protected void onDestroy() {
		super.onDestroy();
		
		logAndAppend(ON_DESTROY);
	}

	@Override
	protected void onSaveInstanceState(Bundle outState) {
		super.onSaveInstanceState(outState);
		logAndAppend(ON_SAVE_INSTANCE_STATE);
		String lifecycleDisplayTextViewContents = mLifecycleDisplay.getText().toString();
		outState.putString(LIFECYCLE_CALLBACKS_TEXT_KEY, lifecycleDisplayTextViewContents);
	}
	
	/**
	* Logs to the console and appends the lifecycle method name to the TextView so that you can
	* view the series of method callbacks that are called both from the app and from within
	* Android Studio's Logcat.
	*
	* @param lifecycleEvent The name of the event to be logged.
	*/
	private void logAndAppend(String lifecycleEvent) {
		Log.d(TAG, "Lifecycle Event: " + lifecycleEvent);
		
		mLifecycleDisplay.append(lifecycleEvent + "\n");
	}
	
	/**
	* This method resets the contents of the TextView to its default text of "Lifecycle callbacks"
	*
	* @param view The View that was clicked. In this case, it is the Button from our layout.
	*/
	public void resetLifecycleDisplay(View view) {
		mLifecycleDisplay.setText("Lifecycle callbacks:\n");
	}
}
\end{lstlisting}

\href{https://developer.android.com/guide/components/activities/activity-lifecycle.html}{Android LifeCycle Resource}

\end{document}
