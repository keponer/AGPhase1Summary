%%%%%%%%%%%%%%%%%%%%%%%%%%%%%%%%%%%%%%%%%
% Simple Sectioned Essay Template
% LaTeX Template
%
% This template has been downloaded from:
% http://www.latextemplates.com
%
% Note:
% The \lipsum[#] commands throughout this template generate dummy text
% to fill the template out. These commands should all be removed when 
% writing essay content.
%
%%%%%%%%%%%%%%%%%%%%%%%%%%%%%%%%%%%%%%%%%

%----------------------------------------------------------------------------------------
%	PACKAGES AND OTHER DOCUMENT CONFIGURATIONS
%----------------------------------------------------------------------------------------

\documentclass[12pt]{article} % Default font size is 12pt, it can be changed here

\usepackage{geometry} % Required to change the page size to A4
\geometry{a4paper} % Set the page size to be A4 as opposed to the default US Letter

\usepackage{graphicx} % Required for including pictures

\usepackage{float} % Allows putting an [H] in \begin{figure} to specify the exact location of the figure
\usepackage{wrapfig} % Allows in-line images such as the example fish picture

\usepackage{lipsum} % Used for inserting dummy 'Lorem ipsum' text into the template

\usepackage{hyperref}


\linespread{1.2} % Line spacing

%\setlength\parindent{0pt} % Uncomment to remove all indentation from paragraphs

\graphicspath{{Pictures/}} % Specifies the directory where pictures are stored

\usepackage{listings}
\usepackage{color}

\definecolor{dkgreen}{rgb}{0,0.6,0}
\definecolor{gray}{rgb}{0.5,0.5,0.5}
\definecolor{mauve}{rgb}{0.58,0,0.82}


\definecolor{Maroon}{rgb}{0.5,0,0}
\definecolor{darkgreen}{rgb}{0,0.5,0}

\lstset{frame=tb,
	backgroundcolor=\color[rgb]{0.9,0.9,0.9},
	language=Java,
	aboveskip=3mm,
	belowskip=3mm,
	showstringspaces=false,
	columns=flexible,
	basicstyle={\small\ttfamily},
	% Line numbers
	xleftmargin={0.75cm},
	numbers=left,
	stepnumber=1,
	firstnumber=1,
	numberfirstline=true,
	numberstyle=\tiny\color{gray},
	keywordstyle=\color{blue},
	commentstyle=\color{dkgreen},
	stringstyle=\color{mauve},
	breaklines=true,
	breakatwhitespace=true,
	tabsize=3
}


\lstdefinelanguage{XML_android}
{
	alsoletter=-,
	basicstyle=\ttfamily\footnotesize,
	morestring=[b]",
	moredelim=*[s][\color{Maroon}]{<}{\ },
	moredelim=[s][\color{Maroon}]{</}{>},
	moredelim=[l][\color{Maroon}]{/>},
	moredelim=[l][\color{Maroon}]{>},
	morecomment=[s]{<?}{?>},
	morecomment=[s]{<!--}{-->},
	morecomment=[s]{<!}{>},
	commentstyle=\color{darkgreen},
	stringstyle=\color{blue},
	identifierstyle=\color{red}
}




\begin{document}

%----------------------------------------------------------------------------------------
%	TITLE PAGE
%----------------------------------------------------------------------------------------

\begin{titlepage}

\newcommand{\HRule}{\rule{\linewidth}{0.5mm}} % Defines a new command for the horizontal lines, change thickness here

\center % Center everything on the page

\HRule \\[0.4cm]
{ \huge \bfseries Create a method to return an url with a GET request}\\[0.4cm] % Title of your document
\HRule \\[1.5cm]

\begin{minipage}{0.4\textwidth}
\end{minipage}\\[4cm]

{\large \today}\\[3cm] % Date, change the \today to a set date if you want to be precise

%\includegraphics{Logo}\\[1cm] % Include a department/university logo - this will require the graphicx package

\vfill % Fill the rest of the page with whitespace

\end{titlepage}

%----------------------------------------------------------------------------------------
%	TABLE OF CONTENTS
%----------------------------------------------------------------------------------------

\tableofcontents % Include a table of contents

\newpage % Begins the essay on a new page instead of on the same page as the table of contents 

%----------------------------------------------------------------------------------------
%	INTRODUCTION
%----------------------------------------------------------------------------------------

\section{Introduction} % Major section

How to create a method which will return a url with a GET request.

%------------------------------------------------

\section{Create the method}

\begin{enumerate}
	\item Create a varible Uri to storage the url.
	\item Use the method parse() to parse your String to uri.
	\item Use the method appendQueryParameter() to add the parameter of your GET request
	\item Use the method build() to build your uri.
	\item build your url from your uri.
	\item surround it with try and catch.
\end{enumerate}

\begin{lstlisting}[language=Java]
public static URL buildUrl(String githubSearchQuery) {
	Uri buildUri = Uri.parse(GITHUB_BASE_URL).buildUpon()
					.appendQueryParameter(PARAM_QUERY, githubSearchQuery)
					.appendQueryParameter(PARAM_SORT, sortBy)
					.build();

	URL url = null;

	try{
		url = new URL(buildUri.toString());
	}catch (MalformedURLException e){
		e.printStackTrace();
	}

	return url;
}
\end{lstlisting}

\href{https://developer.android.com/reference/android/net/Uri.html}{URI API}

\end{document}
