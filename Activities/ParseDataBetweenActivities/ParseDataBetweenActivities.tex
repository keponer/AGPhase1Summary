%%%%%%%%%%%%%%%%%%%%%%%%%%%%%%%%%%%%%%%%%
% Simple Sectioned Essay Template
% LaTeX Template
%
% This template has been downloaded from:
% http://www.latextemplates.com
%
% Note:
% The \lipsum[#] commands throughout this template generate dummy text
% to fill the template out. These commands should all be removed when 
% writing essay content.
%
%%%%%%%%%%%%%%%%%%%%%%%%%%%%%%%%%%%%%%%%%

%----------------------------------------------------------------------------------------
%	PACKAGES AND OTHER DOCUMENT CONFIGURATIONS
%----------------------------------------------------------------------------------------

\documentclass[12pt]{article} % Default font size is 12pt, it can be changed here

\usepackage{geometry} % Required to change the page size to A4
\geometry{a4paper} % Set the page size to be A4 as opposed to the default US Letter

\usepackage{graphicx} % Required for including pictures

\usepackage{float} % Allows putting an [H] in \begin{figure} to specify the exact location of the figure
\usepackage{wrapfig} % Allows in-line images such as the example fish picture

\usepackage{lipsum} % Used for inserting dummy 'Lorem ipsum' text into the template

\usepackage{hyperref}


\linespread{1.2} % Line spacing

%\setlength\parindent{0pt} % Uncomment to remove all indentation from paragraphs

\graphicspath{{Pictures/}} % Specifies the directory where pictures are stored

\usepackage{listings}
\usepackage{color}

\definecolor{dkgreen}{rgb}{0,0.6,0}
\definecolor{gray}{rgb}{0.5,0.5,0.5}
\definecolor{mauve}{rgb}{0.58,0,0.82}


\definecolor{Maroon}{rgb}{0.5,0,0}
\definecolor{darkgreen}{rgb}{0,0.5,0}

\lstset{frame=tb,
	backgroundcolor=\color[rgb]{0.9,0.9,0.9},
	language=Java,
	aboveskip=3mm,
	belowskip=3mm,
	showstringspaces=false,
	columns=flexible,
	basicstyle={\small\ttfamily},
	% Line numbers
	xleftmargin={0.75cm},
	numbers=left,
	stepnumber=1,
	firstnumber=1,
	numberfirstline=true,
	numberstyle=\tiny\color{gray},
	keywordstyle=\color{blue},
	commentstyle=\color{dkgreen},
	stringstyle=\color{mauve},
	breaklines=true,
	breakatwhitespace=true,
	tabsize=3
}


\lstdefinelanguage{XML_android}
{
	alsoletter=-,
	basicstyle=\ttfamily\footnotesize,
	morestring=[b]",
	moredelim=*[s][\color{Maroon}]{<}{\ },
	moredelim=[s][\color{Maroon}]{</}{>},
	moredelim=[l][\color{Maroon}]{/>},
	moredelim=[l][\color{Maroon}]{>},
	morecomment=[s]{<?}{?>},
	morecomment=[s]{<!--}{-->},
	morecomment=[s]{<!}{>},
	commentstyle=\color{darkgreen},
	stringstyle=\color{blue},
	identifierstyle=\color{red}
}




\begin{document}

%----------------------------------------------------------------------------------------
%	TITLE PAGE
%----------------------------------------------------------------------------------------

\begin{titlepage}

\newcommand{\HRule}{\rule{\linewidth}{0.5mm}} % Defines a new command for the horizontal lines, change thickness here

\center % Center everything on the page

\HRule \\[0.4cm]
{ \huge \bfseries Parsing data between activities}\\[0.4cm] % Title of your document
\HRule \\[1.5cm]

\begin{minipage}{0.4\textwidth}
\end{minipage}\\[4cm]

{\large \today}\\[3cm] % Date, change the \today to a set date if you want to be precise

%\includegraphics{Logo}\\[1cm] % Include a department/university logo - this will require the graphicx package

\vfill % Fill the rest of the page with whitespace

\end{titlepage}

%----------------------------------------------------------------------------------------
%	TABLE OF CONTENTS
%----------------------------------------------------------------------------------------

\tableofcontents % Include a table of contents

\newpage % Begins the essay on a new page instead of on the same page as the table of contents 

%----------------------------------------------------------------------------------------
%	INTRODUCTION
%----------------------------------------------------------------------------------------

\section{Introduction} % Major section

How to parse data between activities in android App.

%------------------------------------------------

\section{The main activity}

\begin{enumerate}
	\item Create an intent to the other activity.
	\item Use intent.putExtra(key, value) to send a value and a key to get it to the other activity.
\end{enumerate}

\begin{lstlisting}[language=Java]
// COMPLETED (2) Use the putExtra method to put the String from the EditText in the Intent
/*
* We use the putExtra method of the Intent class to pass some extra stuff to the
* Activity that we are starting. Generally, this data is quite simple, such as
* a String or a number. However, there are ways to pass more complex objects.
*/
startChildActivityIntent.putExtra(Intent.EXTRA_TEXT, textEntered);

\end{lstlisting}

\section{The child class}

\begin{enumerate}
	\item Use the getIntent method to store the Intent that started this Activity in a variable.
	\item Use intent.putExtra(key, value) to send a value and a key to get it to the other activity.
	\item  Create an if statement to check if this Intent has the extra we passed from MainActivity
	\item If the Intent contains the correct extra, retrieve the text.
\end{enumerate}

\begin{lstlisting}[language=Java]
 /*
* Here is where all the magic happens. The getIntent method will give us the Intent that
* started this particular Activity.
*/
Intent intentThatStartedThisActivity = getIntent();

/*
* Although there is always an Intent that starts any particular Activity, we can't
* guarantee that the extra we are looking for was passed as well. Because of that, we need
* to check to see if the Intent has the extra that we specified when we created the
* Intent that we use to start this Activity. Note that this extra may not be present in
* the Intent if this Activity was started by any other method.
* */
if (intentThatStartedThisActivity.hasExtra(Intent.EXTRA_TEXT)) {

	/*
	* Now that we've checked to make sure the extra we are looking for is contained within
	* the Intent, we can extract the extra. To do that, we simply call the getStringExtra
	* method on the Intent. There are various other get*Extra methods you can call for
	* different types of data. Please feel free to explore those yourself.
	*/
	String textEntered = intentThatStartedThisActivity.getStringExtra(Intent.EXTRA_TEXT);
	
}
\end{lstlisting}

\href{https://developer.android.com/reference/android/content/Intent.html}{Intent Resource}

\end{document}
